\addcontentsline{toc}{section}{СПИСОК ИСПОЛЬЗОВАННЫХ ИСТОЧНИКОВ}

\begin{thebibliography}{9}

    \bibitem{1} Фленов, М.Е. Библия C\#. 5-е издание. / М.Е. Фленов. – СПб. : БХВ, 2022. – 464 с. – ISBN 978-5-9775-6827-2. – Текст: непосредственный.
    \bibitem{2} Троелсен, Э. Язык программирования C\# 9 и платформа .NET 5: основные принципы и практики программирования, 10-е издание. / Э. Троелсен, Ф. Джепикс. – Москва: Диалектика, 2022. – 1392 с. – ISBN 978-5-907458-67-3. – Текст: непосредственный.
    \bibitem{3} Умрихин, Е.Д. Разработка веб-приложений с помощью ASP.Net Core MVC. / Е.Д. Умрихин. – СПб. : БХВ, 2023. – 416 с. – ISBN 978-5-9775-1206-0. – Текст: непосредственный.
    \bibitem{4}	Лок, Э. ASP.NET CORE в действии / Э. Лок. – Москва: ДМК Пресс, 2021. – 906 с. – ISBN 978-5-97060-550-9. – Текст: непосредственный.
	\bibitem{5}	Смит, Д. Entity Framework Core в действии / Д. Смит. – Москва: ДМК Пресс, 2023. – 690 с. – ISBN 978-5-93700-114-6. – Текст: непосредственный.
	\bibitem{6}	Фримен, А. Entity Framework Core 2 для ASP.NET Core MVC для профессионалов / А. Фримен. – Москва: Диалектика, 2019. – 624 с. – ISBN 978-5-907114-86-9. – Текст: непосредственный.
	\bibitem{7}	Шилдс, У. SQL: быстрое погружение / У. Шилдс. – СПб. : Питер, 2022. – 224 с. – ISBN 978-5-4461-1835-9. – Текст: непосредственный.
	\bibitem{uchiru}	Болье, А. Изучаем SQL. Генерация, выборка и обработка данных / А. Болье. – Москва: Диалектика-Вильямс, 2021. – 400 с. – ISBN 978-5-907365-54-4. – Текст: непосредственный.
	\bibitem{8}	Осипов, Д.Л. Технологии проектирования баз данных / Д.Л. Осипов. – М.: ДМК Пресс, 2019. – 498 с. – ISBN 978-5-97060-737-4. – Текст: непосредственный.
	\bibitem{22}	Уотсон, Б. Высокопроизводительный код на платформе .NET. 2-е издание / Б. Уотсон. – СПб: Питер, 2020. – 416 с. – ISBN 978-5-4461-0911-1. – Текст: непосредственный.
	\bibitem{9}	Литвиненко, Н.А. Программирование на С\# для платформы .NET Core 3. Курс лекций / Н.А. Литвиненко. – Москва: Горячая Линия - Телеком, 2021. – 328 с. – ISBN 978-5-9912-0874-1. – Текст: непосредственный.
	\bibitem{10}	Черткова, Е.А. Программная инженерия. Визуальное моделирование программных систем / Е.А. Черткова. – Москва: ЮРАЙТ, 2022. – 148 с. – ISBN 978-5-534-09823-5. – Текст: непосредственный.
	\bibitem{11}	Хориков, В. Принципы юнит-тестирования / В. Хориков. – СПб. : Питер, 2022. – 320 с. – ISBN 978-5-4461-1683-6. – Текст: непосредственный.
	\bibitem{14} Маурисио, А. Эффективное тестирование программного обеспечения / А. Маурисио. – Москва: ДМК Пресс, 2023. – 370 с. – ISBN 978-5-97060-997-2. – Текст: непосредственный.
	\bibitem{15} Игнатьев, А. В. Тестирование программного обеспечения / А. В. Игнатьев. – Москва: Лань, 2021. – 56 с. – ISBN 978-5-8114-8072-2. – Текст: непосредственный.
	\bibitem{16} Плаксин, М. А. Тестирование и отладка программ для профессионалов будущих и настоящих / М. А. Плаксин. – Москва: БИНОМ, 2020. – 170 с. – ISBN 978-5-00101-810-0. – Текст: непосредственный.
	\bibitem{17} Ганди, Р. Head First. Git / Р. Ганди. – СПб. : БХВ, 2023. – 464 с. – ISBN 978-5-9775-1777-5. – Текст: непосредственный.
	\bibitem{18} Чакон, C. Git для профессионального программиста / C. Чакон. – Санкт-Петербург: Питер, 2016. – 496 с. – ISBN 978-5-496-01763-3. – Текст: непосредственный.
	\bibitem{19} Фримен, А. ASP.NET MVC 5 с примерами на C\# 5.0 для профессионалов / А. Фримен. – Москва: Вильямс, 2018. – 736 c. – ISBN: 978-5-8459-1911-3. – Текст: непосредственный.
	\bibitem{20} Дронов, В.А. JavaScript. 20 уроков для начинающих. / В.А. Дронов. – СПб. : БХВ, 2020. – 352 с. – ISBN 978-5-9775-6589-9. – Текст: непосредственный.
	\bibitem{21} Дронов, В.А. JavaScript. Дополнительные уроки для начинающих. / В.А. Дронов. – СПб. : БХВ, 2021. – 352 с. – ISBN 978-5-9775-6781-7. – Текст: непосредственный.
	\bibitem{22} Джепикс, Ф. Язык программирования C\# 7 и платформы .NET и .NET Core / Ф. Джепикс, Э. Троелсен. – Москва: Вильямс, 2018. – 1328 c. – ISBN: 978-1-4842-3017-6. – Текст: непосредственный.
	\bibitem{23} Роберт, А. UML для простых смертных / А. Роберт. – Москва: Манн, Иванов и Фербер, 2014. – 272 с. – ISBN 978-5-85582-367-7. – Текст: непосредственный.
	\bibitem{24} Пайлон, Д. UML 2 для программистов / Д. Пайлон. – СПб: Питер, 2012. – 240 с. – ISBN 978-5-459-01684-0. – Текст: непосредственный.
	\bibitem{25} Ларман, К. Применение UML и шаблонов проектирования. Введение в объектно-ориентированный анализ, проектирование и итеративную разработку: учебник и практикум для бакалавриата и магистратуры / К. Ларман. – М.: ООО “И.Д. Вильямс”, 2013. – 426 с. – ISBN 978-5-8459-1185-8.. – Текст: непосредственный.
	\bibitem{26} Гаско, Р. Объектно Ориентированное Программирование / Р. Гаско. – Москва: Солон-Пресс, 2021. – 298 с. – ISBN 978-5-91359-285-9. – Текст: непосредственный.
	\bibitem{27} Вайсфельд, М. Объектно-ориентированное мышление / М. Вайсфельд. – СПб: Питер, 2014. – 304 с. – ISBN 978-5-496-00793-1. – Текст: непосредственный.
	\bibitem{28} Джонсон, Р. Приемы объектно-ориентированного проектирования. Паттерны проектирования / Р. Джонсон, Г. Эрих, Р. Хелм, Д. Влисседес. – Санкт-Петербург: Питер, 2016. – 366 с. – ISBN 978-5-459-01720-5. – Текст: непосредственный.
	\bibitem{29} Коннолли, Т. Базы данных. Проектирование, реализация и сопровождение / Т. Коннолли, К. Бегг. – Москва: Логосфера, 2021. – 1448 с. – ISBN 978-5-94774-185-0. – Текст: непосредственный.
	\bibitem{30} Маккарт, Дж. Чистый код. Создание, анализ и рефакторинг / Дж. Маккарт. – Москва: Диалектика, 2020. – 464 с. – ISBN 978-5-907114-85-2. – Текст: непосредственный.
	\bibitem{31} Бло, Дж. Чистая архитектура. Искусство разработки программного обеспечения / Дж. Бло. – Москва: Питер, 2019. – 432 с. – ISBN 978-5-496-02482-2. – Текст: непосредственный.
	\bibitem{32} Сандерс, Г. Паттерны интеграции корпоративных приложений / Г. Сандерс. – Москва: Бином, 2020. – 368 с. – ISBN 978-5-001-01304-0. – Текст: непосредственный.
	\bibitem{33} Смит, Д. Enterprise Integration Patterns / Д. Смит. – Москва: ДМК Пресс, 2018. – 528 с. – ISBN 978-5-94074-808-4. – Текст: непосредственный.
	\bibitem{34} Джордж, Р. Архитектура корпоративных приложений. Паттерны проектирования / Р. Джордж. – Москва: Логосфера, 2020. – 488 с. – ISBN 978-5-94774-204-8. – Текст: непосредственный.
	\bibitem{35} Голов, А. Микросервисы. Паттерны проектирования / А. Голов. – СПб. : БХВ, 2023. – 544 с. – ISBN 978-5-9775-1347-0. – Текст: непосредственный.
	\bibitem{36} Рихтер, Д. CLR via C\#. Программирование на платформе Microsoft .NET Framework 4.5 на языке C\#. / Д. Рихтер. – М.: Вильямс, 2016. – 896 с. – ISBN 978-5-8459-1913-7. – Текст: непосредственный.
	\bibitem{37} Джонсон, Р. Spring в действии / Р. Джонсон, Д. Хоулер, М. Лайт, Т. Фишер. – М.: Вильямс, 2019. – 688 с. – ISBN 978-5-8459-1917-5. – Текст: непосредственный.
	\bibitem{38} Шапошников, К. Программирование на языке C\#. Полное руководство / К. Шапошников. – СПб.: БХВ-Петербург, 2020. – 768 с. – ISBN 978-5-9775-6906-4. – Текст: непосредственный.
	\bibitem{39} Блошин, А. Эффективный C\#: 50 специфичных рекомендаций для улучшения ваших программ на C\# / А. Блошин. – М.: Вильямс, 2018. – 384 с. – ISBN 978-5-8459-1948-9. – Текст: непосредственный.
	\bibitem{40} Мезенцев, Е. Основы разработки ASP.NET Core 5 / Е. Мезенцев. – СПб.: БХВ-Петербург, 2021. – 352 с. – ISBN 978-5-9775-7891-2. – Текст: непосредственный.
	\bibitem{41} Гудман, Д. JavaScript jQuery. Исчерпывающее руководство / Д. Гудман. – СПб.: Питер, 2019. – 1072 с. – ISBN 978-5-496-01342-0. – Текст: непосредственный.
	\bibitem{42} Албахари, Дж. C\# 8.0 и .NET Core 3.0. Карманный справочник / Дж. Албахари, Б. Албахари. – СПб.: Питер, 2020. – 1040 с. – ISBN 978-5-4461-1464-1. – Текст: непосредственный.
	\bibitem{43} Петцольд, Ч. Программирование для Windows на языке C\#. / Ч. Петцольд. – М.: Вильямс, 2021. – 1104 с. – ISBN 978-5-8459-1905-2. – Текст: непосредственный.
	\bibitem{44} Кроули, М. Эффективное управление базами данных. / М. Кроули. – СПб.: Питер, 2019. – 672 с. – ISBN 978-5-496-01601-8. – Текст: непосредственный.
	\bibitem{45} Харгривз, Д. Разработка приложений на платформе Microsoft .NET. / Д. Харгривз. – СПб.: БХВ-Петербург, 2018. – 512 с. – ISBN 978-5-9775-7102-8. – Текст: непосредственный.
	\bibitem{46} Фримен, А. Программирование ASP.NET MVC 4 с примерами на C\# 5.0 / А. Фримен. – М.: Вильямс, 2017. – 832 с. – ISBN 978-5-8459-1832-1. – Текст: непосредственный.
	\bibitem{47} Бертон, Б. Программирование на C\#. Полное руководство / Б. Бертон. – СПб.: Питер, 2021. – 864 с. – ISBN 978-5-4461-1878-6. – Текст: непосредственный.
	\bibitem{48} Криспин, Л. Гибкое тестирование: как сделать тестирование ценным для всех / Л. Криспин, Д. Грегори. – М.: Вильямс, 2020. – 456 с. – ISBN 978-5-8459-1958-8. – Текст: непосредственный.
	\bibitem{49} Шрайбер, Р. Архитектура программного обеспечения: от концепций к практике / Р. Шрайбер. – М.: Вильямс, 2019. – 368 с. – ISBN 978-5-8459-1916-8. – Текст: непосредственный.
	\bibitem{50} Куусела, Р. Разработка корпоративных приложений на платформе .NET / Р. Куусела. – СПб.: БХВ-Петербург, 2020. – 576 с. – ISBN 978-5-9775-7643-6. – Текст: непосредственный.

\end{thebibliography}
