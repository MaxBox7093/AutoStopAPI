\section{Анализ предметной области}
\subsection{Общие принципы и понятия поиска попутчиков для поездки}

Поиск попутчиков для поездки — это концепция, направленная на оптимизацию транспортных ресурсов путем совместного использования транспортных средств несколькими людьми, имеющими схожие маршруты и временные рамки. Эта практика получила новый импульс благодаря развитию цифровых технологий и увеличению осведомленности об экологических проблемах.

Принципы поиска попутчиков:

\begin{itemize}
\item Экономическая выгода. Совместные поездки позволяют участникам делить расходы, что делает путешествия более доступными. В условиях постоянно растущих цен на топливо и обслуживание автомобилей возможность снизить затраты становится важным преимуществом.

\item Экологическая устойчивость. Совместные поездки способствуют снижению количества автомобилей на дорогах, что уменьшает выбросы вредных веществ в атмосферу. Это важный аспект в контексте глобальной борьбы с изменением климата.

\item Социальные аспекты. Поиск попутчиков способствует укреплению социальных связей и взаимодействию между людьми. Социальные контакты, установленные в процессе совместных поездок, могут перерасти в крепкие дружеские отношения.

\item Гибкость и адаптивность. Совместные поездки предоставляют участникам возможность выбирать наиболее удобные для них маршруты и время отправления, делая процесс путешествия более гибким и адаптивным к потребностям каждого.
\end{itemize}

Понятия, связанные с поиском попутчиков:

\begin{itemize}
\item Совместное использование транспорта. Основная концепция, подразумевающая использование одного транспортного средства несколькими людьми, что позволяет оптимизировать затраты и уменьшить нагрузку на дорожную инфраструктуру.

\item Маршруты и расписание. Важным аспектом является согласование маршрутов и времени отправления между участниками. Это требует координации и планирования для обеспечения максимального удобства для всех попутчиков.

\item Экономия на расходах. Деление расходов на топливо и обслуживание транспортного средства является значительным стимулом для участия в совместных поездках.

\item Социальные взаимодействия. В процессе совместных поездок люди имеют возможность общаться, делиться опытом и устанавливать новые социальные связи.
\end{itemize}

Преимущества поиска попутчиков:

\begin{itemize}
\item Снижение нагрузки на дороги. Меньшее количество автомобилей на дорогах способствует уменьшению заторов и улучшению дорожной ситуации.

\item Улучшение экологической обстановки. Уменьшение выбросов вредных веществ способствует улучшению качества воздуха и снижению негативного воздействия на окружающую среду.

\item Повышение социальной активности. Поиск попутчиков способствует развитию социальных контактов и взаимодействий, что может быть полезно для личностного роста и создания новых дружеских связей.
\end{itemize}

Вызовы и трудности

\begin{itemize}
\item Координация участников. Согласование маршрутов и времени отправления может быть сложным процессом, особенно если участники имеют разные графики и предпочтения. Это требует гибкости и готовности к компромиссам.

\item Непредвиденные обстоятельства. Изменения планов одного из участников могут повлиять на весь маршрут, требуя перестройки планов и поиска новых попутчиков.

\item Доверие и безопасность. Одним из ключевых факторов успешного поиска попутчиков является доверие между участниками. Важно, чтобы все участники чувствовали себя комфортно и безопасно в процессе совместных поездок.
\end{itemize}

\subsection{Приложения для поиска попутчиков, их классификация}

Приложения для поиска попутчиков -- это цифровые платформы, предназначенные для объединения людей, которые хотят совершить совместные поездки. Эти приложения служат посредниками между водителями и пассажирами, предоставляя удобный интерфейс для поиска, бронирования и планирования поездок.

Основная цель приложений для поиска попутчиков -- упростить процесс организации совместных поездок, обеспечивая комфорт и экономичность для всех участников. Водители, имеющие свободные места в автомобиле, могут размещать информацию о предстоящих поездках, а пассажиры, нуждающиеся в транспорте, могут искать и бронировать подходящие поездки.

Основные функции приложений для поиска попутчиков:

\begin{itemize}
	\item Регистрация и создание профиля. Пользователи регистрируются в приложении, создают свои профили, указывая личную информацию и предпочтения по поездкам. Это позволяет другим участникам ознакомиться с профилем пользователя перед поездкой.
	
	\item Поиск и бронирование поездок. Водители размещают объявления с информацией о маршруте, времени отправления и количестве свободных мест. Пассажиры могут искать подходящие поездки, используя фильтры по различным параметрам, таким как местоположение, дата и время отправления, цена и предпочтения по уровню комфорта.
	
	\item Коммуникация между пользователями. Встроенные мессенджеры позволяют водителям и пассажирам уточнять детали поездки, договариваться о времени и месте встречи, а также обмениваться необходимой информацией.
	
	\item Оплата поездок. Приложения могут предлагать различные способы оплаты, включая предоплату через электронные платежные системы или оплату наличными по факту поездки. Это обеспечивает гибкость и удобство для пользователей.
\end{itemize}

Преимущества приложений для поиска попутчиков:

\begin{itemize}
	\item Быстрота поиска. В системах приложений уже есть размещенные поездки, которые попутчики могут забронировать в любую секунду, что ускоряет процесс поиска и бронирования.
	
	\item Удобство. Приложения предоставляют пользователям удобный интерфейс для поиска и бронирования поездок, что экономит время и усилия.
	
	\item Гибкость. Пользователи могут выбирать маршруты и время отправления, которые лучше всего соответствуют их потребностям и предпочтениям.
\end{itemize}

Модели взаимодействия.
Приложения для поиска попутчиков могут использовать различные модели взаимодействия между водителями и пассажирами:

\begin{itemize}
	\item Долгосрочные поездки. Пользователи могут находить попутчиков для регулярных и длительных поездок, таких как поездки на работу или учебу.
	
	\item Краткосрочные поездки. Приложения также позволяют находить попутчиков для разовых поездок на короткие расстояния, например, для поездок на мероприятия или в соседние города.
	
	\item Специальные маршруты. Некоторые приложения предлагают специализированные маршруты, такие как трансферы до аэропортов или курортов, что позволяет пользователям планировать свои поездки заранее.
\end{itemize}

Разновидности приложений.
Существует несколько разновидностей приложений для поиска попутчиков в зависимости от их специализации и модели взаимодействия с пользователями:

\begin{itemize}
	\item Городские поездки. Приложения, ориентированные на совместные поездки в пределах города, помогают разгрузить транспортную инфраструктуру и снизить заторы.
	
	\item Междугородные поездки. Эти приложения фокусируются на длительных поездках между городами, предоставляя пользователям возможность экономить на транспорте.
	
	\item Тематика путешествий. Некоторые приложения специализируются на поездках для определенных целей, таких как путешествия, туризм или деловые поездки.
\end{itemize}

Приложения для поиска попутчиков играют важную роль в современной транспортной экосистеме, предлагая удобные, экономичные и гибкие решения для совместных поездок. Они способствуют улучшению качества жизни пользователей, снижению транспортных расходов и оптимизации использования транспортных ресурсов.

\subsection{Исследование аналогов}
\subsubsection{BlablaCar}
BlablaСar – Приложение позиционирует себя как инструмент для поиска попутчиков. Является лидером на рынке среди приложений в данной сфере. Позволяет связываться с попутчиками через личные сообщения внутри приложения, а также через средства мобильной связи, после подтверждения поездки.

Основные плюсы данного аналога:

\begin{itemize}
\item Популярный сервис для поиска попутчиков. Является приложением лидером в данной сфере и практически не имеет конкурентов.
\item Богатый функционал приложения.
\item Продуманный дизайн.
\item Имеет встроенный мессенджер для обсуждения планов поездки.
\item Большая база попутчиков и водителей.
\item Возможность бронирования поездки заранее.
\item Позволяет искать попутчиков по системе «Буст», когда водитель может взять попутчика, проездом.
\item Имеет систему рейтинга.
\item Понятный сайт.
\item Имеет приложение на OC андроид.
\item Встроенная интеграция с поиском автобусов.
\item Возможность ставить точки на карте, для уточнения места встречи попутчика и водителя.
\item Имеет поддержку пользователя.
\end{itemize}


Минусами сервиса BlaBlacar являются следующие критерии:

\begin{itemize}
\item Возможны проблемы при использовании приложения на слабых android устройствах.
\item Сервис платный. При использовании приложения, может взыматься плата при поездках между населенными пунктами на расстояние в более чем 100км.
\item Не имеет desktop версию на Windows и Mac OS.
\end{itemize}

\subsubsection{Едем.рф}
Еще одним приложением для поиска попутчиков, является сервис Едем.рф. Данный сервис тоже пользуется большой популярностью и во многом схож с главный конкурентом – BlablaCar. Особенностью данного приложения является реализации грузоперевозок межгород.
Приложение позволяет связываться с попутчиками 2 способами:

\begin{itemize}
\item Через внутренний чат приложения.
\item Через средства мобильной связи в формате звонка.
\end{itemize}

Плюсы данного приложения:

\begin{itemize}
\item Популярный сервис для поиска попутчиков.
\item Продуманный дизайн приложения.
\item Имеет встроенный мессенджер для обсуждения планов поездки.
\item Большая база попутчиков и водителей.
\item Возможность бронирования поездки заранее.
\item Имеет систему рейтинга.
\item Встроенная интеграция с поиском автобусов.
\item Возможность ставить точки на карте, для уточнения точки встречи попутчика и водителя.
\item Имеет поддержку пользователя.
\item Система отзывов и рейтингов.
\end{itemize}

Минусами сервиса едем.рф являются следующие критерии:

\begin{itemize}
\item Приложение имеет некоторые баги.
\item Сервис платный. При использовании приложения взымается плата для водителей, при создании объявления о поездке.
\item Не имеет desktop версию на Windows и Mac
\end{itemize}

\subsubsection{Попутка.рус}

Последним аналогом в данной предметной области является приложение Попутка.рус. Данное приложения не пользуется большой популярностью. Так же приложение имеет возможность публикации объявлений с предоставлением контактных данных. 

Приложение позволяет поддерживать диалог между попутчиком и водителем 3 способами:

\begin{itemize}
\item Телефонная связь.
\item Внутренний мессенджер внутри приложения.
\item Сообщения в формате sms на телефон.
\end{itemize}

Приложение имеет следующие плюсы:

\begin{itemize}
\item Есть возможность создавать объявления, в том числе и для осуществления сделки купли/продажи.
\item Функционал для связи водителя и попутчиков.
\item Простое приложение, которое имеет малый вес и без проблем может работать даже на слабых устройствах.
\item Есть возможность авторизации и создания аккаунта с привязкой только к номеру телефона.
\item В процессе поиска водителя позволяет выбрать категорию поездки.
\end{itemize}

Минусы приложения Попутка.рус:

\begin{itemize}
\item Маленькая база пользователей. Найти попутчика проблематично.
\item Функционал данного приложения в плане поиска попутчиков уступает аналогам.
\item Менее привлекательный интерфейс.
\end{itemize}

\subsection{Пользовательское исследование}
В данном разделе описаны варианты использования приложения для поиска попутчиков, в зависимости от требований пользователей и различных ситуаций. Был проведен опрос, в котором пользователи подробно описали свой вариант использования приложения для достижения определенных целей, в виду особенностей и требованиям к системе.

Первый вариант использования приложения:

Мужчина среднего возраста планирует дальнюю поездку на своем автомобиле. Ему необходимо найти попутчиков для того, чтобы поездка на была слишком накладной. Мужчина до этого не использовал приложение, и планирует набрать небольшую базу попутчиков. Автомобиль: renault logan.

Он заходит в play market и скачивает приложение, проходит простую регистрацию, в ходе которой указывает свой номер телефона и личную информацию. Далее в самом приложении мужчина находит раздел размещения поездки. Система понимает, что у данного водителя еще не было осуществлено поездок и предлагает пройти дополнительную регистрацию для водителей, в ходе которой мужчина регистрирует свой автомобиль, указывает опыт вождения и дополнительную информацию, необходимую для осуществления поездки.

После прохождения всех этапов регистрации, мужчина размещает поездку через вышеупомянутый раздел. В качестве информации, он указывает точку, откуда он собирается начать свой путь, конечную точку маршрута, дату и время поездки. А также, указывает автомобиль, на котором собирается осуществить поездку.

Так как у данного водителя небольшой опыт использования приложения, водитель указывает цену поездки чуть ниже рыночной. Через некоторое время, водителю удается найти несколько попутчиков, которые звонят на его контактный номер телефона и обсуждают планы проведения поездки.

Попутчики и водитель встречаются в указанной точке в соответствии со временем начала поездки. После успешного проведения поездки, водитель получает денежные средства от попутчиков.

Второй вариант использования приложения:

Молодому парню требуется передать посылку для матери, которая находится в другом городе. У него нет необходимости в передачи посылки из рук в руки. Он решает воспользоваться приложением AutoStop. Для осуществления передачи.

Парень заходит в приложение и осуществляет поиск водителя через соответствующий раздел. В качестве информации, он указывает город отправления, конченую точку маршрута и дату начала поездки. В результате чего находит несколько водителей. Далее парень осуществляет выборку водителей.

Для осуществления передачи посылки парню не требуется бронировать место в автомобиле. Вместо этого он решает вести диалог через сообщения внутри приложения с водителем. В ходе диалога, водитель договаривается с парнем о передачи посылки за соответствующую плату. И обменивается контактными данными для осуществления передачи.
Далее они встречаются на месте начальной точки маршрута, и парень передает посылку водителю. Водитель, в свою очередь, по окончанию поездки вручает посылку в руки матери молодого человека и получает за это плату.

Третий вариант использования приложения:

Супружеская пара почтенного возраста планирует добраться из точки А в точку Б. Для них использования приложения вызывает множество трудностей в силу возраста. Вместо бронирования мест, для осуществления поездки со своего телефона, они просят свою дочь забронировать им места.

Для бронирования поездки, дочка входит в приложение и осуществляет подбор водителей в соответствующем разделе. Она тщательно просматривает водителей в соответствии с указанными критериями. Одним из таких является комфортный автомобиль. 

После недолгих поисков, она находит водителя Михаила с достаточно большим стажем вождения на Volvo XC90. Понимая, что автомобиль может обеспечить комфортную поездку её родителям, она указывает в качестве брони 2 места. И пишет водителю о том, что в качестве попутчиков, будут её родители. На что Михаил соглашается.
После чего, у девушки появляется возможность позвонить Михаилу и обсудить тонкости поездки. В ходе обсуждения узнается, что у водителя полностью заполнен багажник и в салоне автомобиля будут ехать еще 3 человека. Такой вариант, не устраивает девушку, и она отменяет бронь на данную поездку.

В процессе поиска, девушка находит водителя с достаточным стажем и комфортным автомобилем. В чате с водителем она обсуждает все тонкости и в итоге они договариваются о поездке.

Четвертый вариант использования приложения:

Женщина среднего возраста планирует деловую поездку в другой город и решает воспользоваться приложением для поиска попутчиков на своем компьютере с операционной системой Windows. Ей важно найти надежного водителя и комфортный автомобиль для поездки.

Женщина запускает приложение AutoStop, которое уже установлено на её компьютере. Она проходит стандартную процедуру регистрации, указав свою личную информацию и номер телефона. После успешной регистрации она переходит в раздел поиска поездок.

В интерфейсе поиска она указывает точку отправления, конечный пункт назначения, дату поездки. Система отображает список доступных водителей, которые соответствуют её запросу.

После тщательного изучения она выбирает водителя с комфортным автомобилем, Audi A6. Женщина отправляет запрос на бронирование места, указав, что она путешествует одна и у нее будет небольшой багаж.

Женщина обсуждает детали поездки через встроенный чат в приложении. Они договариваются о месте встречи.
