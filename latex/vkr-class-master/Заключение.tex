\section*{ЗАКЛЮЧЕНИЕ}
\addcontentsline{toc}{section}{ЗАКЛЮЧЕНИЕ}

Разработка серверной части кроссплатформенного приложения для поиска попутчиков является актуальной и перспективной задачей. Современный рынок показывает растущий интерес к приложениям для совместных поездок, что обусловлено экономическими и экологическими факторами, а также удобством использования данных сервисов. В процессе работы были изучены потребности и ожидания пользователей, конкурентная ситуация и особенности технологической реализации серверной части приложения.

Основные результаты работы:
\begin{enumerate}
	\item Проведен анализ предметной области, что позволило выявить ключевые требования и ожидания пользователей от подобных приложений, а также определить основные конкурентные решения и их функциональные особенности.
	\item Разработана концептуальная модель программно-информационной системы, включающая варианты использования системы и требования к серверной части.
	\item Спроектирована и реализована серверная часть программной системы. Разработана архитектура системы с описанием основных компонентов и их взаимодействия, а также ключевых классов и методов.
	\item Проведено тестирование разработанной серверной части программной системы. Выполнено функциональное и системное тестирование для обеспечения корректной работы всех компонентов системы.
\end{enumerate}

Все требования, объявленные в техническом задании, были полностью реализованы. Все задачи, поставленные в начале разработки проекта, были также решены.

Предполагается, что разрабатываемая программная система будет использоваться широким кругом лиц, заинтересованных в совместных поездках, включая тех, кто ищет экономичные и экологически чистые способы передвижения.

Перспективы дальнейшей разработки программной системы включают расширение функционала приложения, увеличение числа пользователей, улучшение алгоритмов поиска и предложения попутчиков, а также интеграцию новых сервисов, таких как динамическое ценообразование и дополнительные средства безопасности для пользователей.