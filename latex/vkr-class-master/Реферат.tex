\abstract{РЕФЕРАТ}

Объем работы равен \formbytotal{lastpage}{страниц}{е}{ам}{ам}. Работа содержит \formbytotal{figurecnt}{иллюстраци}{ю}{и}{й}, \formbytotal{tablecnt}{таблиц}{у}{ы}{}, \arabic{bibcount} библиографических источников и \formbytotal{числоПлакатов}{лист}{}{а}{ов} графического материала. Количество приложений – 2. Графический материал представлен в приложении А. Фрагменты исходного кода представлены в приложении Б.

Перечень ключевых слов: кроссплатформенная система, поиск попутчиков, автомобильные поездки, серверная часть, информатизация, автоматизация, информационные технологии, автоматический сервис, классы, база данных, компонент, модуль, сущность, метод, попутчик, пользователь, водитель, маршруты, транспорт.

Объектом разработки является кроссплатформенная программная система для поиска попутчиков для автомобильных поездок.

Целью выпускной квалификационной работы является создание удобного и функционального сервиса для поиска попутчиков, что способствует экономии ресурсов, снижению транспортных расходов и уменьшению нагрузки на окружающую среду.

В процессе разработки системы были выделены основные сущности, такие как пользователи, маршруты и поездки, и созданы соответствующие классы и методы модулей, обеспечивающие взаимодействие с данными сущностями и корректную работу системы. Были разработаны и реализованы функциональные компоненты, такие как регистрация пользователей, создание и поиск маршрутов, бронирование мест в поездках, а также мессенджер.

Для реализации серверной части системы использовались технологии C\#, ASP.net MVC и Microsoft SQL для создания надежной и производительной базы данных. Обеспечена поддержка различных платформ и устройств благодаря адаптивному дизайну и оптимизации запросов к серверу.

\selectlanguage{english}
\abstract{ABSTRACT}
  
The volume of work is \formbytotal{lastpage}{page}{}{s}{s}. The work contains \formbytotal{figurecnt}{illustration}{}{s}{s}, \formbytotal{tablecnt}{table}{}{s}{s}, \arabic{bibcount} bibliographic sources and \formbytotal{числоПлакатов}{sheet}{}{s}{s} of graphic material. The number of applications is 2. The graphic material is presented in annex A. The layout of the site, including the connection of components, is presented in annex B.
The list of keywords: cross-platform system, travel companion search, car trips, backend, informatization, automation, information technology, automatic service, classes, database, component, module, entity, method, travel companion, user, driver, routes, transport.

The object of the development is a cross-platform software system for finding travel companions for car trips.

The purpose of the final qualification work is to create a convenient and functional service for finding travel companions, which helps to save resources, reduce transport costs and reduce the burden on the environment.

During the development of the system, the main entities such as users, routes and trips were identified, and appropriate classes and methods of modules were created to ensure interaction with these entities and the correct operation of the system. Functional components have been developed and implemented, such as user registration, route creation and search, travel reservations, and messenger.

C\# technologies were used to implement the server part of the system, ASP.net MVC and Microsoft SQL to create a reliable and productive database. Support for various platforms and devices is provided thanks to adaptive design and optimization

\selectlanguage{russian}
