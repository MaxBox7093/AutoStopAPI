\section*{ВВЕДЕНИЕ}
\addcontentsline{toc}{section}{ВВЕДЕНИЕ}

Сфера приложений для подбора попутчиков набирает популярность и становится важной частью современной транспортной экосистемы. С каждым годом все больше людей предпочитают использовать онлайн-сервисы для поиска попутчиков, что приводит к снижению затрат на поездки и уменьшению нагрузки на дороги. Благодаря развитию информационных технологий, процесс поиска и бронирования поездок стал удобным и доступным для всех пользователей.

Пользователи таких приложений получают возможность быстро найти попутчиков, ознакомиться с их профилями и выбрать оптимальный маршрут и удобное время отправления. Водители, в свою очередь, могут заполнить свободные места в своем автомобиле, сократив расходы на топливо и автомобильные расходы. Это создает взаимовыгодные условия для обеих сторон.

Одним из ключевых факторов успешного функционирования приложений для подбора попутчиков является удобство использования и функциональность. Современные приложения предлагают пользователям встроенный мессенджер для быстрого и удобного общения, возможность создавать и управлять аккаунтами, а также инструменты для планирования маршрута и выбора попутчиков. Эти функции значительно упрощают процесс поиска и бронирования поездок.

В условиях высокой конкуренции на рынке транспортных приложений, разработчики стремятся обеспечить кроссплатформенную доступность своих сервисов. Это позволяет пользователям независимо от используемого устройства получать доступ ко всем функциям приложения и наслаждаться комфортом использования.

Особое внимание уделяется удобству интерфейса и качеству предоставляемых сервисов. Современные приложения используют интуитивно понятные интерфейсы и предлагают пользователям разнообразные функции для удобства планирования поездок, такие как автоматическое определение оптимального маршрута и расчет времени в пути.

Однако, несмотря на все преимущества, пользователи могут столкнуться с некоторыми трудностями. Например, в пиковые часы спрос на поездки может превышать предложение, что приводит к задержкам и увеличению стоимости. Кроме того, важно учитывать географические особенности и наличие инфраструктуры для удобного доступа к месту встречи с водителем.

В целом, мобильные приложения для подбора попутчиков представляют собой инновационное решение, способствующее развитию устойчивого и эффективного транспорта. Благодаря им пользователи могут экономить время и деньги, а также вносить свой вклад в снижение выбросов CO2 и улучшение экологической ситуации.


\emph{Цель настоящей работы} – разработка серверной части кроссплатформенного приложения для поиска попутчиков. Для достижения поставленной цели необходимо решить \emph{следующие задачи:}
\begin{itemize}
\item провести анализ предметной области;
\item разработать концептуальную модель программно-информационной системы;
\item спроектировать и реализовать серверную часть программной системы;
\item провести тестирование серверной части программной системы.
\end{itemize}

\emph{Структура и объем работы.} Отчет состоит из введения, 4 разделов основной части, заключения, списка использованных источников, 2 приложений. Текст выпускной квалификационной работы равен \formbytotal{lastpage}{страниц}{е}{ам}{ам}.

\emph{Во введении} сформулирована цель работы, поставлены задачи разработки, описана структура работы, приведено краткое содержание каждого из разделов.

\emph{В резюме стартап-проекта} содержится основная информация о стартап-проекте: название, цели и стратегия, уникальность продукта, результаты, риски и перспективы проекта

\emph{В первом разделе} на стадии описания технической характеристики предметной области приводится сбор информации об аналогах, а так же производится пользовательское исследование, на основе которого осуществляется разработка программной системы.

\emph{Во втором разделе} на стадии технического задания приводятся требования к серверной части.

\emph{В третьем разделе} на стадии технического проектирования представлены проектные решения для разрабатываемого приложения.

\emph{В четвертом разделе} приводится список классов и их методов, использованных при разработке API, производится тестирование разработанного решения.

В заключении излагаются основные результаты работы, полученные в ходе разработки.

В приложении А представлен графический материал.
В приложении Б представлены фрагменты исходного кода. 
