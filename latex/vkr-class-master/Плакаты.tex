\appendix{Представление графического материала}

Графический материал, выполненный на отдельных листах,
изображен на рисунках А.1--А.\arabic{числоПлакатов}.
\setcounter{числоПлакатов}{0}

\renewcommand{\thefigure}{А.\arabic{figure}} % шаблон номера для плакатов

\begin{landscape}

\begin{плакат}
	\includegraphics[width=0.82\linewidth]{images/Placat1}
	\заголовок{Сведения о ВКРБ}
	\label{pl1:placat1}
\end{плакат}

\begin{плакат}
	\includegraphics[width=0.82\linewidth]{images/Placat2}
	\заголовок{Цель и задачи разработки}
	\label{pl1:placat2}
\end{плакат}

\begin{плакат}
	\includegraphics[width=0.82\linewidth]{images/Placat3}
	\заголовок{Диаграммы вариантов использования}
	\label{pl1:placat3}
\end{плакат}

\begin{плакат}
	\includegraphics[width=0.82\linewidth]{images/Placat4}
	\заголовок{ER - Диаграмма базы данных}
	\label{pl1:placat4}
\end{плакат}

\begin{плакат}
	\includegraphics[width=0.82\linewidth]{images/Placat5}
	\заголовок{Диаграмма развертывания программы}
	\label{pl1:placat5}
\end{плакат}

\begin{плакат}
	\includegraphics[width=0.82\linewidth]{images/Placat6}
	\заголовок{Диаграммы классов моделей и контроллеров}
	\label{pl1:placat6}
\end{плакат}

\begin{плакат}
	\includegraphics[width=0.82\linewidth]{images/Placat7}
	\заголовок{Диаграмма классов SQL запросов}
	\label{pl1:placat7}
\end{плакат}

\begin{плакат}
	\includegraphics[width=0.82\linewidth]{images/Placat8}
	\заголовок{Заключение}
	\label{pl1:placat8}
\end{плакат}

\end{landscape}
