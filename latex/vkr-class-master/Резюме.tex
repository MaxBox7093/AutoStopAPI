\section*{РЕЗЮМЕ СТАРТАП-ПРОЕКТА}
\addcontentsline{toc}{section}{РЕЗЮМЕ СТАРТАП-ПРОЕКТА}

Название: «Бизнес-проект «Кроссплатформенная программная система поиска попутчиков для автомобильных поездок»».
Кроссплатформенное приложение для поиска попутчиков AutoStop -- это программа, где все пользователи делятся на попутчиков и водителей. Водитель может создавать поезки и размещать их внутри приложения, в свою очередь, попутчики могут находить поездки и бронировать их. В приложении можно выстраивать коммуникацию между пользователями для обсуждения планов поездки через чаты.

Были проанализированы современные аналоги и выявлены основные проблемы, с которыми сталкиваются пользователи приложений поиска попутчиков для автомобильных поездок:

\begin{enumerate}
	\item Высокая стоимость сервисного сбора и услуг, которые предоставляют приложения для поиска попутчиков. Основная часть приложений данной сферы, запрашивает сервисный сбор, который может достигать 30\% от цены, которую указал водитель. Это побуждает пользователей действовать в обход системы, снимая бронь с поездки. Эти действия создают сложности для комфортного использования подобных сервисов.
	\item Отсутствие desktop версий под операционные системы Mac os и Windows. Большинство сервисов имеет web сайты для бронирования поездок, однако этот подход создает уязвимость для неопытных пользователей, которые ошибочно могут зайти на сайт мошенников.
	\item Некоторые сервисы имеют проблемы с оптимизацией и плохо адаптированы под слабые устройства.
	\item Низкая пользовательская база. Так как практически все приложения для поиска попутчиков требуют сервисный сбор, база пользователей не растет. Для приложения, специализирующегося на автомобильных поездках, очень важно иметь активных пользователей. В противном случае, использование сервиса становится невозможным.
\end{enumerate}

Кроссплатформенная программная система поиска попутчиков для автомобильных поездок AutoStop призвана решить эти проблемы, и стать удобным сервисом для взаимодействия попутчиков и водителей.

Актуальность данного сервиса связана с растущей потребностью в удобных и экономичных способах организации совместных автомобильных поездок. На сегодняшний день на рынке отсутствуют кроссплатформенные решения, которые бы полностью удовлетворяли потребности пользователей. Существующие приложения часто обременены высокими сервисными сборами и имеют ограниченные функциональные возможности. AutoStop решает эти проблемы, предлагая доступный и удобный сервис для водителей и попутчиков, который поддерживает популярные операционные системы и устройства.

Отличительные особенности программной системы по сравнению с существующими решениями и технологиями:

\begin{enumerate}
	\item На данном этапе весь функционал приложения предоставляется бесплатно. Это решение направлено на привлечение потенциальных пользователей и быстрое расширение базы активных пользователей для обеспечения успешного функционирования сервиса в будущем.
	\item AutoStop поддерживает популярные операционные системы и устройства, включая Mac OS, Windows, и Android. Это обеспечивает пользователям максимальную гибкость и удобство при использовании сервиса, позволяя им бронировать поездки и общаться с другими участниками независимо от используемого устройства.
\end{enumerate}

В разработанном приложении реализованы следующие возможности:

Регистрация пользователей при помощи номера телефона под аккаунтом водителя или попутчика, изменения данных профиля, добавление автомобилей под учетной записью водителя, смена фотографии профиля, создание поездки под аккаунтом водителя, удаление созданной поездки, просмотр архива пользовательских поездок, поиск поездки, бронирование активных поездок, отмена бронирования, встроенный мессенджер позволяет поддерживать связь между водителем и попутчиком.

Идея создания проекта «AutoStop» возникла в январе 2023 года, однако разработка началась позже в связи с принятием решений, касательно концепции работы приложения. С июля 2024 года планируется расширение функционала приложения и развертывание на общедоступном сервере для тестирования работы приложения под нагрузкой, на этом этапе принимать участие в тестировании смогут пользователи по всей стране, с октября 2024 планируется полноценное оказание услуг. Оказание услуг не зависит от региона Российской Федерации.

Для защиты исключительных прав и правовой охраны необходимо в дальнейшем осуществить регистрацию программного средства.

Самыми значительными и высоковероятными рисками для проекта «AutoStop» являются приложения аналоги, у которых есть схожий функционал и большая аудитория пользователей. Перспективой дальнейшей разработки программной системы является наращивание аудитории, увеличение количества поездок, размещенных на сервере, а также расширение функционала посредством обновлений приложения.

Бизнес-цели:

\begin{enumerate}
\item Увеличить число зарегистрированных пользователей: 5000 пользователей через год. Для достижения этой цели можно использовать различные методы привлечения пользователей, например, рекламные кампании в
социальных сетях, поисковую оптимизацию и т.д.
\item Увеличить колличество поддерживаемых платформ до максимума, добавив в поддержку устройства на базе платформы IOS.
\item Добавить систему отзывов и рейтингов. Для повышения надежности и уверенности в успешном исходе каждой поездки.
\item Обеспечить надежную и бесперебойную работу серверной части. Необходимо провести исследование и расчеты рисков разных вариантов хостинга.
\item Выйти на стабильную прибыль через 2 года. Для достижения этой цели можно использовать различные методы монетизации системы, например, рекламу, введение комиссии за дальние поездки, и т.д.
\end{enumerate}